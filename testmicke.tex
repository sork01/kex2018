% *neode.onsave* setgo rm seminardraft.pdf; xelatex -interaction=nonstopmode seminardraft; biber seminardraft; xelatex -interaction=nonstopmode seminardraft; xelatex -interaction=nonstopmode seminardraft; evince seminardraft.pdf

\documentclass[bachelor]{kththesis}

\usepackage{csquotes} % Recommended by biblatex
\usepackage{biblatex}
\addbibresource{sources.bib} % The file containing our references, in BibTeX format

\usepackage{color}

\newcommand{\blue}[1]{{\color{blue} [#1]}}
\newcommand{\red}[1]{{\color{red} \{#1\}}}

\title{Local search hybridization of a genetic algorithm for solving the University Course Timetabling Problem}
\alttitle{Lokalsökningshybridisering av en genetisk algoritm som löser schemaläggningsproblemet UCTP}
\author{Mikael Forsberg}
\email{miforsb@kth.se}
\supervisor{Alexander Kozlov}
\examiner{Örjan Ekeberg}
\programme{Bachelor's degree in Computer Science}
\school{School of Computer Science}
\date{\today}


\begin{document}

% Frontmatter includes the titlepage, abstracts and table-of-contents
\frontmatter

\titlepage

\begin{abstract}
  English abstract goes here.
\end{abstract}


\begin{otherlanguage}{swedish}
  \begin{abstract}
    Abstrakt på svenska.
  \end{abstract}
\end{otherlanguage}


\tableofcontents


% Mainmatter is where the actual contents of the thesis goes
\mainmatter


\chapter{Introduction}
\noindent \blue{Text in angle brackets is descriptions of planned content (descriptive)}

\noindent \red{Text in curly brackets is sketches of planned content (illustrative)}

\vspace{1cm}

\noindent \blue{Less-technical introductory paragraph.}
% We use the \emph{biblatex} package to handle our references.  We therefore use the
% command \texttt{parencite} to get a reference in parenthesis, like this
% \parencite{heisenberg2015}.  It is also possible to include the author
% as part of the sentence using \texttt{textcite}, like talking about
% the work of \textcite{einstein2016}.

The University Course Timetabling Problem (UCTP) is a scheduling problem in which a set of events
(lectures, labs) are to be assigned locations (lecture halls, computer rooms) and time slots (time and date)
while satisfying a number of constraints such as avoiding double-booking of locations, ensuring that
events are assigned to locations equipped with suitable features (projectors, fume hoods) and minimizing
the number of events overlapping in time that have the same participants (lecturers, student groups).

\red{There are many variants of UCTP ...}

More formally, \red{most real-world applicable variants of?} UCTP is an NP-hard\parencite{lovelace} combinatorial optimization problem, meaning the % TODO: reformulate, UCTP is a combinatorial optimization problem that is NP-hard in most nontrivial instances (or something similar)
search for optimal solutions to non-trivial problem instances is currently thought to be generally
intractable, turning the attention to methods of approximation. Many methods of approximation fall
under the umbrella term of metaheuristics\parencite{lewis}, which is a collection of algorithmic ideas
often inspired by natural processes. Examples of metaheuristics include evolutionary algorithms,
simulated annealing and ant colony optimization. The current state of the art lies in hybrid
metaheuristics where a population-based method is combined with some form of local
search\parencite{ilyas-iqbal}.

\section{Previous research}
\begin{itemize}
\item \blue{Super brief history of UCTP}
\item \red{Yamazaki-Pertoft Genetic Algorithm\parencite{yamazaki-pertoft}}
\item \red{Renman-Fristedt Tabu Search\parencite{renman-fristedt}}
\end{itemize}

\section{Problem statement}
\begin{itemize}
\item \red{Which hybridization strategy yields the most effective UCTP solver?}
\item \red{Can hybridization improve scalability?}
\end{itemize}

\chapter{Background}
\section{The University Course Timetabling Problem}
\blue{More depth on UCTP including types of constraints}

\subsection{Formal description}
\blue{Formal description of the UCTP variant being solved by algorithms in this work - does this belong here? or under Method?}

\section{Metaheuristics}
\blue{Brief history and description of metaheuristics}

\subsection{Genetic Algorithms}
\blue{General description followed by pseudocode for the genetic algorithm used}

\subsection{Tabu Search}
\blue{Description and pseudocode for Tabu Search}

\section{Hybridization strategies}
\blue{Brief thoughts on how there are many different strategies}
\subsection{Switchover}
\blue{Describing the basic idea of switching algorithms at some threshold fitness}

\subsection{Interleaving}
\blue{Describing the basic idea of alternating between the two algorithms and substrategies in 
only alternating at certain thresholds}

\red{Definition: to arrange (an operation) so that two or more programs, sets of instructions, etc., are performed in an alternating fashion. - Dictionary.com}

\chapter{Method}
\section{Implementation}
\blue{Description of the specific algorithm variants / hybridization strategies actually used in the experiment}
\section{Experiment}
\blue{Description of the YP testcases used (KTH\_M, KTH\_L etc))}

\chapter{Results}
\section{Preliminary results}
\bgroup
\renewcommand{\arraystretch}{1.5}
\begin{tabular}{|l l r r|}
\hline
Testcase & Algorithm & Trials & Average (seconds)\\
\hline
KTH_M & YPGA & 10 & 6.8 s\\
KTH_M & Interleave & 10 & 5.7 s\\
\hline
KTH_L & YPGA & 10 & 52.9 s\\
KTH_L & Interleave & 10 & 17.9 s\\
\hline
\end{tabular}
\egroup

\chapter{Discussion}
\section{Improvements}
\begin{itemize}
\item \red{Use more improved candidates from TS when updating the GA}
\item \red{Parallelization - run both GA and TS at the same time, when one finds an improvement, update the other one}
\end{itemize}

\chapter{Conclusion}

\printbibliography[heading=bibintoc] % Print the bibliography (and make it appear in the table of contents)

\appendix

\chapter{Unnecessary Appended Material}

\end{document}