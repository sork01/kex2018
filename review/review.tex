% *neode.onsave* setgo pdflatex -interaction=nonstopmode review; bibtex review; pdflatex -interaction=nonstopmode review; pdflatex -interaction=nonstopmode review; evince review.pdf

\documentclass[a4paper,12pt]{article}
\usepackage[margin=3.5cm]{geometry}
\linespread{1}
\usepackage[T1]{fontenc}
\usepackage[utf8]{inputenc}
\usepackage{lmodern}
\usepackage{amsmath}
\usepackage{amsfonts}
\usepackage{amssymb}
\usepackage{graphicx}
\linespread{1.25}

\begin{document}
\noindent
\small KTH-CSC/DD142X
% \title{\vspace{-1cm} Peer review protocol \vspace{.3cm} }
% \author{Mikael Forsberg <miforsb@kth.se>}
% {\let\newpage\relax\maketitle}

\begin{center}
\huge Peer review protocol
\end{center}

\noindent
\textbf{Authors:} Martin Hyberg and Teodor Hurtigh Isaacs


\noindent
\textbf{Report:} Sentiment Analysis

\noindent
\textbf{Reviewer:} Robin Gunning <Rgunning@kth.se>

\noindent
\textbf{Date:} May 23, 2018

\section{Summary of the report}
The authors have studied sentiment analysis, which in short is to classify a piece of text as either positive or negative. This is supposed to be done on 50 different youtube videos chosen based on specific criteria which is not presented at the time of this draft. 
The comments on the youtube-video will be used to ''predict'' if the ''like-ratio'' will be positive or negative. 

The authors presents a few different ways to handle this task but there is no section yet explaining which method the authors will use to get their results.


The authors have a placeholder image in the result section but there is not much information about what the results mean. My guess is that the result is non conclusive and you can't really predict the ''like-ratio''


\section{Title and abstract}
The title is not very informative and seems like it is incomplete. Sentiment Analysis could mean anything. At this point there is no abstract to comment on.



\section{Introduction and problem statement}
The introduction is fine, but some wording problems exist such as: 

\begin{quote}
''Thanks to social media everyone with a network connection has the opportunity to
voice their opinions at a global scale regarding current events
worldwide".
\end{quote}

\noindent Global and world wide have essentially the same meaning.\\
The problem statement is too short and just scarecly touches on the problem at hand, the problem statement should contain the use of machine learning or it might aswell ask if someone manually can guess the like-ratio from reading the comments


\section{Background and related work}
The authors write:

\begin{quote}
    In this thesis this will be done using three
    different machine learning algorithms to classify each comment into
    two categories, positive or negative.
\end{quote}

\noindent
However there is no information about what three algorithms that will be used, I am guessing it's ''Support Vector Machine'', ''Logistic Regression'' and ''Naïve Bayes''. This needs to be stated when you write about the three different algorithms 

\subsection{Neural Networks (ANN)}
Perhaps the section should be named ''Artificial Neural Networks'' to match the given ANN acronym.
Moreover, the section feels extremely abstract and non-technical. Perhaps you could include a
typical textbook example of a basic ANN.

Section numbering would be helpful in showing that the following sections on CNN and RNN/LSTM
are actually subsection to the ANN section.

\section{Methods}
\subsection{Data}
I'm not sure I know what is meant by ''The data was preprocessed using absolute fast fourier
transform''. Perhaps include some background on EEG data and its (pre)processing. Maybe this is
something that is painfully obvious to someone familiar with processing EEG data (which I am not),
in which case I guess it could be left as is.

\subsection{Model selection - SVM} \label{strangesentence}
The following sentence is strange: ''In the input data for the models there are 16 frames with
summarized samples for beta and mu for the C3 and C4 channels respectively there are in total 64
features at this point.''

\subsection{Model selection - LSTM}
You write ''For creating the LSTM we used Keras [...]''. You could perhaps mention this ''Keras'' in
the background or atleast include a footnote with a link. Including specific version numbers might also be
appropriate.

\subsection{Model selection - CNN}
\label{similarly}
You write ''Similarly the CNN model was created [...]''. Similarly to what?

\section{Results}
Maybe I missed it, but I don't remember seeing a description of how you selected data for / trained the
''model trained on all data'' in the Method chapter.

The descriptions of ANOVA and hypothesis testing found at the end of the Results chapter should be
moved to the Method chapter. ANOVA should be capitalized.

\subsection{SVM}
The number of digits in ''Accuracy of model trained on all subjects 0.5251141552511416'' seems
excessive.

\section{Discussion and conclusions}
\subsection{Results}
You write:

\begin{quote}
    [...] three [sic] seemed to be a trend of high inter-subject and inter-session accuracy for
    subject s1, s2, low for subject s3 and s4. This could be taken to imply that some individuals
    produce EEG data more suitable for generalization than others and vice versa [...]
\end{quote}

\label{strangethink}
\noindent
Besides the initial typo (''three'', should be ''there''), this seems to me a strange way of
thinking about things. If the models can generalize over
certain groups of people but not others, then surely it is the case that the models are
not good enough for achieving a general solution, not that some individuals just produce
unsuitable data. It seems more reasonable to say something like ''This could be taken
to imply that there exists subsets of people that certain models are able to generalize over''.

\section{Overall characteristics}
% Overall characteristics (coherence, presentation style, structure, language)
Disposition structure and language are both fine. Coherence is only slightly
broken in a couple of places where ambiguous or undefined terms are used; e.g the first uses of the
terms ''performance'' and ''accuracy'', the use of ''single-subject BCI classifier'', the
use of ''Keras'', potentially the use of ''extracranial'' depending on the intended audience. Overall
coherence is fine.

The style could be improved by adding chapter and section numbering and by printing page numbers
on each page. Tables in the Method and Results chapters should be numbered. Graphs in the results
chapter should be made into numbered figures. Figures 3, 4 and 5 seem not to be referred to in the
text.

\section{Overall impression}
The strongest points are structure and language. The paper is easy to read and to follow, meaning
the essential parts are all mostly present. The weakest points are the unnumbered figures and tables.

\section{Suggestions}
These are the suggestions made throughout this document, collected in a bullet point list
for convenience.

\begin{itemize}
    \item Style
    \begin{itemize}
        \item Add chapter and section numbering.
        \item Print page number on each page.
        \item After numbering all figures (e.g graphs) and titling + numbering all tables, make sure
        to reference each figure and each table in the text atleast once.
        \item Capitalize ANOVA.
    \end{itemize}
    \item Abstract
    \begin{itemize}
        \item Re-formulate the statement ''Our results show that inter-subject generalization depends on the subject''.
        \item Name the factors affecting generalization ability and your suggested future avenues of research.
    \end{itemize}
    \item Introduction and problem statement
    \begin{itemize}
        \item Clearly define the term ''single-subject classifier''.
        \item Clearly define the terms ''performance'' and ''accuracy'' in your context.
        \item Consider adapting the problem statement to the results, it seems you are not really investigating the stated specific relationship.
    \end{itemize}
    \item Background
    \begin{itemize}
        \item Re-formulate or expand the paragraph on how users are required to learn to control
        a given BCI such that the idea of a generalized system becomes less contradictory.
        \item Rename the section ''Neural Networks (ANN)'' to ''Artificial Neural Networks (ANN)'' such
        that the acronym matches.
        \item Add some brief technical details on ANNs.
        \item Add a section on EEG data and its preprocessing using absolute fast fourier transform.
    \end{itemize}
    \item Method
    \begin{itemize}
        \item Number and title all tables.
        \item Include a clear description of how the data was selected for the ''model trained on all data''.
        \item Fix a strange sentence (see \ref{strangesentence})
        \item Include a footnote with a link to Keras
        \item Re-formulate the leading sentence in the section on CNN model selection (see \ref{similarly})
    \end{itemize}
    \item Results
    \begin{itemize}
        \item Have all graphs be numbered figures.
        \item Number and title all tables.
        \item Move the descriptions of ANOVA and hypothesis testing to the Method chapter.
        \item Re-formulate the leading sentence in the section on CNN model selection (see \ref{similarly})
        \item Reduce or motivate the large number of digits in the result for the SVM model trained on all subjects.
    \end{itemize}
    \item Discussion and conclusion
    \begin{itemize}
        \item Fix a typo ''three seemed'' -> ''there seemed''.
        \item Consider re-formulating the parts on how some people seem to generate unsuitable data (see \ref{strangethink})
        \item Reduce or motivate the large number of digits in the result for the SVM model trained on all subjects.
    \end{itemize}
\end{itemize}

\end{document}